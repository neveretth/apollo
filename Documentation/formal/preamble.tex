\documentclass[]{article}

% ====== FONT ======
\usepackage{fontspec}
% \setmainfont{EBGaramond-Regular.ttf}
\newfontfamily{\berkeleymono}[Path=./font/]{BerkeleyMono-Regular.ttf}

% ====== PAGE SIZE & MARGINS ======
\usepackage{geometry}
% \geometry{a4paper, left=32mm, right=32mm, top=36mm, bottom=36mm}
\geometry{letterpaper, left=28mm, right=28mm, top=36mm, bottom=36mm}

% ====== SETTINGS ======
\pagenumbering{arabic}

% ====== IN-DOCUMENT SETTINGS ======
% These settings have to be dont this way because they need to come after \begin{document}
\newcommand{\settings}[0]{
% ====== BEGIN INTERNAL SETTINGS ======
\fontsize{10.4pt}{14.8pt}\selectfont
% ====== END INTERNAL SETTINGS ======
}

% ====== STANDARD PACKAGE IMPORTS ======
\usepackage{amsmath}
\usepackage{fancybox}
\usepackage{float}
\usepackage{mathtools}
\usepackage{enumitem}
\setlist{noitemsep, topsep=6pt, parsep=1pt, partopsep=0pt}

% Equation tags (1.1.2) <-- this thing on the right.
\makeatletter
% \newtagform{sectiontag}[\thesubsection\alph{equation}\@gobble]{(}{)}
\newtagform{sectiontag}[\numberwithin{equation}{subsection}]{(}{)}
\makeatother

% ====== CODE ======
\usepackage{minted}
% These look like this.
% \inputminted[frame=lines, linenos, firstline=3, lastline=6]{c}{example.c}
%
% Inline, but actual code (ish). (minted)
% \begin{minted}[frame=lines, linenos]{c}
% int func(int* type) {
%   for (int i = 0; i < SIZE; i++) {
%     type[i] *= FLUX;
%   }
% }
% \end{minted}
% ====== LISTINGS ====== 
% This is also for code, I'm not sure which one I hate more. 
% The intention is for the maximum width of this box to be 80COL.
% In truth it's more like 79...
\usepackage{listings}
\lstset{
    frame=tbrl,
    tabsize=4,
    showstringspaces=false,
    % numbers=left,
    basicstyle=\berkeleymono\footnotesize,
    xleftmargin=2mm,
    xrightmargin=2mm,
} 
\lstnewenvironment{code}{}{}


% ====== GRAPHICS ======
\usepackage{graphicx}
\graphicspath{ {./img/} }
\usepackage{pgfplots}
\usepackage{pgfplotstable}
\usepackage{circuitikz}

% ====== BIBLIOGRAPHY ======
\usepackage[style=authoryear, backend=biber]{biblatex}
\addbibresource{bibliography.bib}

% ====== HEADER & FOOTER ======
\usepackage{fancyhdr}
\pagestyle{fancy}
\fancyhf{}
% L -> left | C -> center | R -> right 
% O -> odd | E -> even <== This doesn't seem to work properly
\fancyhead[L]{}
\fancyhead[C]{}
\fancyhead[R]{}
\fancyfoot[L]{APOLLO Contributors}
\fancyfoot[C]{\thepage}
\fancyfoot[R]{University of Tennessee}
\renewcommand{\headrulewidth}{0.25pt}
\renewcommand{\footrulewidth}{0.25pt}
\fancypagestyle{titlepage}{
  \fancyfoot[L]{APOLLO Contributors}
  \fancyfoot[C]{\thepage}
  \fancyfoot[R]{University of Tennessee}
}
