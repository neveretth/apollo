\documentclass[]{article}

% ====== FONT ======
\usepackage{fontspec}
% \setmainfont{EBGaramond-Regular.ttf}
\newfontfamily{\berkeleymono}[Path=./font/]{BerkeleyMono-Regular.ttf}

% ====== PAGE SIZE & MARGINS ======
\usepackage{geometry}
% \geometry{a4paper, left=32mm, right=32mm, top=36mm, bottom=36mm}
\geometry{letterpaper, left=28mm, right=28mm, top=36mm, bottom=36mm}

% ====== SETTINGS ======
\pagenumbering{arabic}

% ====== IN-DOCUMENT SETTINGS ======
% These settings have to be dont this way because they need to come after \begin{document}
\newcommand{\settings}[0]{
% ====== BEGIN INTERNAL SETTINGS ======
\fontsize{10.4pt}{14.8pt}\selectfont
% ====== END INTERNAL SETTINGS ======
}

% ====== STANDARD PACKAGE IMPORTS ======
\usepackage{amsmath}
\usepackage{fancybox}
\usepackage{float}
\usepackage{mathtools}
\usepackage{enumitem}
\setlist{noitemsep, topsep=6pt, parsep=1pt, partopsep=0pt}

% Equation tags (1.1.2) <-- this thing on the right.
\makeatletter
% \newtagform{sectiontag}[\thesubsection\alph{equation}\@gobble]{(}{)}
\newtagform{sectiontag}[\numberwithin{equation}{subsection}]{(}{)}
\makeatother

% ====== CODE ======
\usepackage{minted}
% These look like this.
% \inputminted[frame=lines, linenos, firstline=3, lastline=6]{c}{example.c}
%
% Inline, but actual code (ish). (minted)
% \begin{minted}[frame=lines, linenos]{c}
% int func(int* type) {
%   for (int i = 0; i < SIZE; i++) {
%     type[i] *= FLUX;
%   }
% }
% \end{minted}
% ====== LISTINGS ====== 
% This is also for code, I'm not sure which one I hate more. 
% The intention is for the maximum width of this box to be 80COL.
% In truth it's more like 79...
\usepackage{listings}
\lstset{
    frame=tbrl,
    tabsize=4,
    showstringspaces=false,
    % numbers=left,
    basicstyle=\berkeleymono\footnotesize,
    xleftmargin=2mm,
    xrightmargin=2mm,
} 
\lstnewenvironment{code}{}{}


% ====== GRAPHICS ======
\usepackage{graphicx}
\graphicspath{ {./img/} }
\usepackage{pgfplots}
\usepackage{pgfplotstable}
\usepackage{circuitikz}

% ====== BIBLIOGRAPHY ======
\usepackage[style=authoryear, backend=biber]{biblatex}
\addbibresource{bibliography.bib}

% ====== HEADER & FOOTER ======
\usepackage{fancyhdr}
\pagestyle{fancy}
\fancyhf{}
% L -> left | C -> center | R -> right 
% O -> odd | E -> even <== This doesn't seem to work properly
\fancyhead[L]{}
\fancyhead[C]{}
\fancyhead[R]{}
\fancyfoot[L]{APOLLO Contributors}
\fancyfoot[C]{\thepage}
\fancyfoot[R]{University of Tennessee}
\renewcommand{\headrulewidth}{0.25pt}
\renewcommand{\footrulewidth}{0.25pt}
\fancypagestyle{titlepage}{
  \fancyfoot[L]{APOLLO Contributors}
  \fancyfoot[C]{\thepage}
  \fancyfoot[R]{University of Tennessee}
}

\newcommand{\addfig}[4]{
\begin{figure}[H]
\vspace{0mm}
\noindent\minipage{\linewidth}
\begin{center}
  \scalebox{#2}{
    \includegraphics[angle=#3, width=\textwidth]{#1}
  }\\
\end{center}
\endminipage
\vspace{0mm}
\caption{#4}
\vspace{-2mm}
\end{figure}
}

\renewcommand{\table}[4]{
\begin{figure}[H]
\vspace{0mm}
\noindent\minipage{\linewidth}
\begin{center}
  \scalebox{#1}{
    \begin{tabular}{#2}
      #3
    \end{tabular}
  }\\
\end{center}
\endminipage
\vspace{0mm}
\caption{#4}
\vspace{-2mm}
\end{figure}
}

\newcommand{\eq}[1]{
\vspace{-4mm}
\begin{equation}
\begin{split}
  #1
\end{split}
\end{equation}
}

\renewcommand{\l}[0]{
  \left(
}

\renewcommand{\r}[0]{
  \right)
}

\newcommand{\circuit}[2]{
\begin{figure}[H]
\vspace{0mm}
\noindent\minipage{\linewidth}
\begin{center}
  \scalebox{1}{
    \begin{circuitikz}[american voltages] \draw
      #1
    ;
    \end{circuitikz}
  }\\
\end{center}
\endminipage
\vspace{0mm}
\caption{#2}
\vspace{-2mm}
\end{figure}
}

% I cannot fucking _believe_ this is the syntax for hrule.
\renewcommand{\rule}{
  \vspace{4mm}
  \hrule height 0.25pt
  \vspace{4mm}
}


\begin{document}

% Settings that have to be inside the \begin...
\settings
\usetagform{sectiontag}

\title{Formal Outline of APOLLO Software Logic}
\author{\Large{APOLLO Contributors}
\smallskip\\ % <== This shit is so stupid, why LaTeX, why?
% Dr. Michael Guidry\\
% Nicholas Everett Howard\\
% Adam Cole\\
}
\date{\today}
\maketitle

\thispagestyle{titlepage}

% % \hrule
% \vspace{4mm}
% \hrule
% \vspace{4mm}
% \begin{abstract}
% \normalsize
% \end{abstract}

\rule

\tableofcontents

\settings
% \normalsize
\newpage

\section{Understanding This Documentation}

The idea of this documentation is simple. Any physical or mathematical logic
used by the APOLLO code should be documented here. 

\subsection{Example}

To write a formal outline you would first explain
what you are doing and why (where it comes from, sources, etc). Then you would
write out the mathematical logic (so far this is exactly how a normal paper
would do things). Then if the logic is sufficiently complex, or we are
employing some trickery, we would explain how we implement it in code.

Here's a simple equation for an example.

\eq {
  \vec{V} = \vec{V}_x \lambda
}

Pseudocode to explain logic if necessary (only for complex or non-intuitive
things).

\begin{code}
for (elem in V):
  v[idx] = v_x[idx] * lambda
\end{code}

Finally, the \emph{most important thing.} Each equation is given a unique 
number, put that number in a comment above the code that implements the
logic of that equation.

\begin{code}
// [formal 1.1.1]
for (int i = 0; i < 100; i++) {
    // Code that performs the logic...
    // Code that performs the logic...
    // Code that performs the logic...
}
\end{code}

The comment should follow this standard template.

\begin{code}
// [formal <equation>]
// examples
// [formal 1.2.3]
// [formal 2.1.9]
// [formal 1.2.5]
\end{code}



\newpage
\section{Hydrodymanics}
\subsection{Advection}

Lorem ipsum dolor sit amet, qui minim labore adipisicing minim sint cillum sint
consectetur cupidatat.

\eq {
  F = ma
}


\newpage
\section{Thermonuclear Evolution}

Lorem ipsum dolor sit amet, qui minim labore adipisicing minim sint cillum sint
consectetur cupidatat.

\newpage
\section{Neutrino Interaction}

Lorem ipsum dolor sit amet, qui minim labore adipisicing minim sint cillum sint
consectetur cupidatat.

% \begin{code}{language=c}{
% for () {
%   MATH = MATH;
% }
% \end{code}

\newpage
\printbibliography

\end{document}
